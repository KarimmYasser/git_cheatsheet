% Setting up the document class and essential packages
\documentclass[a4paper,10pt]{article}
\usepackage[utf8]{inputenc}
\usepackage[T1]{fontenc}
\usepackage{lmodern}
\usepackage{geometry}
\geometry{margin=0.7in}
\usepackage{hyperref}
\hypersetup{
    colorlinks=true,
    linkcolor=blue,
    urlcolor=blue,
    pdfborder={0 0 0}
}
\usepackage{titling}
\usepackage{tocloft}
\usepackage{enumitem}
\usepackage{parskip}
\setlength{\parskip}{0.5em}
\usepackage{xcolor}
\usepackage{verbatim}
\usepackage{fancyhdr}


\pagestyle{fancy}
\fancyhf{}
\fancyhead[C]{\thetitle}
\fancyfoot[C]{\thepage}
\renewcommand{\headrulewidth}{0.4pt}

% Defining custom commands for consistent formatting
\newcommand{\sectiontitle}[1]{%
    \begin{center}
        \section*{#1}
    \end{center}
    \vspace{-0.5em}
}
\newcommand{\subsectiontitle}[1]{\subsection*{#1}\vspace{-0.3em}}
\newcommand{\cmd}[1]{\texttt{#1}}
\newcommand{\mycomment}[1]{\textit{\# #1}}

% Setting the title
\title{Git Cheat Sheet}
\author{Karim Yasser}
\date{}

\begin{document}

% Title page
\begin{center}
    \vspace*{10cm} % Add vertical space at the top
    {\Huge \textbf{Git Cheat Sheet}} \\[1cm] % Large title with bold
    {\Large The essential Git commands every developer must know} \\[1cm] % Large subtitle
    \vspace{0.5cm}
    \textit{by Karim Yasser} % Author in italic
    \vspace{2cm} % Add space at the bottom
\end{center}
\newpage

% Creating Snapshots
\sectiontitle{Creating Snapshots}
\subsectiontitle{Configuring Git}
\begin{verbatim}
git config --global user.name "Your Name"   # Sets your name for commits
git config --global user.email "your.email@example.com"  # Sets your email for commits
git config --global core.editor "nano"      # Sets the default editor for Git messages
git config --global --list                  # Lists all Git configuration settings
\end{verbatim}

\subsectiontitle{Initializing a repository}
\begin{verbatim}
git init
\end{verbatim}

\subsectiontitle{Staging files}
\begin{verbatim}
git add file1.js                # Stages a single file
git add file1.js file2.js       # Stages multiple files
git add *.js                    # Stages with a pattern
git add .                       # Stages the current directory and all its content
git add -u                      # Stages modified and deleted files, not new files
\end{verbatim}

\subsectiontitle{Ignoring Files}
\begin{verbatim}
# Create or edit a .gitignore file to exclude files from tracking
echo "node_modules/" >> .gitignore  # Ignores the node_modules directory
echo "*.log" >> .gitignore          # Ignores all .log files
git status                          # Verify ignored files are no longer tracked
\end{verbatim}

\subsectiontitle{Viewing the status}
\begin{verbatim}
git status                      # Full status
git status -s                   # Short status
\end{verbatim}

\subsectiontitle{Committing the staged files}
\begin{verbatim}
git commit -m "Message"         # Commits with a one-line message
git commit                      # Opens the default editor to type a long message
git commit --no-edit            # Commits without changing the message during amend
\end{verbatim}

\subsectiontitle{Skipping the staging area}
\begin{verbatim}
git commit -am "Message"
\end{verbatim}

\subsectiontitle{Removing files}
\begin{verbatim}
git rm file1.js                 # Removes from working directory and staging area
git rm --cached file1.js        # Removes from staging area only
\end{verbatim}

\subsectiontitle{Renaming or moving files}
\begin{verbatim}
git mv file1.js file1.txt
\end{verbatim}
\newpage

% Browsing History
\sectiontitle{Browsing History}
\subsectiontitle{Viewing the staged/unstaged changes}
\begin{verbatim}
git diff                        # Shows unstaged changes
git diff --staged               # Shows staged changes
git diff --cached               # Same as the above
git diff --name-only            # Lists only the names of changed files
\end{verbatim}

\subsectiontitle{Viewing the history}
\begin{verbatim}
git log                         # Full history
git log --oneline               # Summary
git log --reverse               # Lists the commits from the oldest to the newest
git log --graph                 # Visualizes branch history with a graph
\end{verbatim}

\subsectiontitle{Viewing a commit}
\begin{verbatim}
git show 921a2ff                # Shows the given commit
git show HEAD                   # Shows the last commit
git show HEAD~2                 # Two steps before the last commit
git show HEAD:file.js           # Shows the version of file.js stored in the last commit
\end{verbatim}

\subsectiontitle{Unstaging files (undoing git add)}
\begin{verbatim}
git restore --staged file.js    # Copies the last version of file.js from repo to index
\end{verbatim}

\subsectiontitle{Discarding local changes}
\begin{verbatim}
git restore file.js             # Copies file.js from index to working directory
git restore file1.js file2.js   # Restores multiple files in working directory
git restore .                   # Discards all local changes (except untracked files)
git clean -fd                   # Removes all untracked files
\end{verbatim}

\subsectiontitle{Restoring an earlier version of a file}
\begin{verbatim}
git restore --source=HEAD~2 file.js
\end{verbatim}

\subsectiontitle{Filtering the history}
\begin{verbatim}
git log --stat                  # Shows the list of modified files
git log --patch                 # Shows the actual changes (patches)
git log -3                      # Shows the last 3 entries
git log --author="Mosh"         # Commits by author "Mosh"
git log --before="2020-08-17"   # Commits before a date
git log --after="one week ago"  # Commits after a date
git log --grep="GUI"            # Commits with "GUI" in their message
git log -S"GUI"                 # Commits with "GUI" in their patches
git log hash1..hash2            # Range of commits
git log file.txt                # Commits that touched file.txt
\end{verbatim}

\subsectiontitle{Formatting the log output}
\begin{verbatim}
git log --pretty=format:"%an committed %H"
\end{verbatim}

\subsectiontitle{Creating an alias}
\begin{verbatim}
git config --global alias.lg "log --oneline"
\end{verbatim}

\subsectiontitle{Comparing commits}
\begin{verbatim}
git diff HEAD~2 HEAD            # Shows the changes between two commits
git diff HEAD~2 HEAD file.txt   # Changes to file.txt only
\end{verbatim}

\subsectiontitle{Checking out a commit}
\begin{verbatim}
git checkout dad47ed            # Checks out the given commit
git checkout master             # Checks out the master branch
\end{verbatim}

\subsectiontitle{Finding a bad commit}
\begin{verbatim}
git bisect start
git bisect bad                  # Marks the current commit as a bad commit
git bisect good ca49180         # Marks the given commit as a good commit
git bisect reset                # Terminates the bisect session
\end{verbatim}

\subsectiontitle{Finding contributors}
\begin{verbatim}
git shortlog
\end{verbatim}

\subsectiontitle{Viewing the history of a file}
\begin{verbatim}
git log file.txt                # Shows the commits that touched file.txt
git log --stat file.txt         # Shows statistics (the number of changes) for file.txt
git log --patch file.txt        # Shows the patches (changes) applied to file.txt
\end{verbatim}

\subsectiontitle{Finding the author of lines}
\begin{verbatim}
git blame file.txt              # Shows the author of each line in file.txt
\end{verbatim}

\subsectiontitle{Tagging}
\begin{verbatim}
git tag v1.0                    # Tags the last commit as v1.0
git tag v1.0 5e7a828           # Tags an earlier commit
git tag                         # Lists all the tags
git tag -d v1.0                 # Deletes the given tag
\end{verbatim}
\newpage

% Branching & Merging
\sectiontitle{Branching \& Merging}
\subsectiontitle{Managing branches}
\begin{verbatim}
git branch bugfix               # Creates a new branch called bugfix
git checkout bugfix             # Switches to the bugfix branch
git switch bugfix               # Same as the above
git switch -C bugfix            # Creates and switches
git branch -d bugfix            # Deletes the bugfix branch
git branch -m bugfix newname    # Renames the branch to newname
\end{verbatim}

\subsectiontitle{Comparing branches}
\begin{verbatim}
git log master..bugfix          # Lists the commits in the bugfix branch not in master
git diff master..bugfix          # Shows the summary of changes
\end{verbatim}

\subsectiontitle{Stashing}
\begin{verbatim}
git stash push -m "New tax rules"  # Creates a new stash
git stash list                  # Lists all the stashes
git stash show stash@{1}        # Shows the given stash
git stash show 1                # Shortcut for stash@{1}
git stash apply 1               # Applies the given stash to the working dir
git stash pop                   # Applies the latest stash and removes it from the list
git stash drop 1                # Deletes the given stash
git stash clear                 # Deletes all the stashes
\end{verbatim}

\subsectiontitle{Merging}
\begin{verbatim}
git merge bugfix                # Merges the bugfix branch into the current branch
git merge --no-ff bugfix        # Creates a merge commit even if FF is possible
git merge --squash bugfix       # Performs a squash merge
git merge --abort               # Aborts the merge
\end{verbatim}

\subsectiontitle{Viewing the merged branches}
\begin{verbatim}
git branch --merged             # Shows the merged branches
git branch --no-merged          # Shows the unmerged branches
\end{verbatim}

\subsectiontitle{Rebasing}
\begin{verbatim}
git rebase master               # Changes the base of the current branch
git rebase --continue           # Continues a rebase after resolving conflicts
\end{verbatim}

\subsectiontitle{Cherry picking}
\begin{verbatim}
git cherry-pick dad47ed         # Applies the given commit on the current branch
\end{verbatim}

\subsectiontitle{Managing Worktrees}
\begin{verbatim}
git worktree add <path> <branch>            # Creates a new worktree at <path> for <branch>
git worktree add ../myapp-feature-login feature/login
git worktree list                           # Lists all linked worktrees
git worktree remove <path>                  # Deletes a worktree
git worktree remove ../myapp-feature-login
git worktree prune                          # Cleans up stale worktree metadata
\end{verbatim}

\newpage

% Collaboration
\sectiontitle{Collaboration}
\subsectiontitle{Cloning a repository}
\begin{verbatim}
git clone url
\end{verbatim}

\subsectiontitle{Syncing with remotes}
\begin{verbatim}
git fetch origin master         # Fetches master from origin
git fetch origin                # Fetches all objects from origin
git fetch                       # Shortcut for "git fetch origin"
git fetch --prune               # Removes stale remote-tracking branches
git pull                        # Fetch + merge
git pull --rebase               # Fetch + rebase instead of merge
git push origin master          # Pushes master to origin
git push                        # Shortcut for "git push origin master"
\end{verbatim}

\subsectiontitle{Sharing tags}
\begin{verbatim}
git push origin v1.0            # Pushes tag v1.0 to origin
git push origin --delete v1.0   # Deletes tag v1.0 from origin
\end{verbatim}

\subsectiontitle{Sharing branches}
\begin{verbatim}
git branch -r                   # Shows remote tracking branches
git branch -vv                  # Shows local & remote tracking branches
git push -u origin bugfix       # Pushes bugfix to origin
git push -d origin bugfix       # Removes bugfix from origin
\end{verbatim}

\subsectiontitle{Managing remotes}
\begin{verbatim}
git remote                      # Shows remote repos
git remote -v                   # Shows remote repos with their URLs
git remote add upstream url     # Adds a new remote called upstream
git remote rename upstream newname  # Renames the remote 'upstream' to 'newname'
git remote rm upstream          # Removes upstream
\end{verbatim}
\newpage

% Rewriting History
\sectiontitle{Rewriting History}
\subsectiontitle{Undoing commits}
\begin{verbatim}
git reset --soft HEAD^          # Removes the last commit, keeps changes staged
git reset --mixed HEAD^         # Unstages the changes as well
git reset --hard HEAD^          # Discards local changes
git reset --keep HEAD^          # Resets to HEAD^, keeps untracked files
\end{verbatim}

\subsectiontitle{Reverting commits}
\begin{verbatim}
git revert 72856ea              # Reverts the given commit
git revert HEAD~3               # Reverts the last three commits
git revert --no-commit HEAD~3   # Reverts without committing
\end{verbatim}

\subsectiontitle{Recovering lost commits}
\begin{verbatim}
git reflog                      # Shows the history of HEAD
git reflog show bugfix          # Shows the history of bugfix pointer
\end{verbatim}

\subsectiontitle{Amending the last commit}
\begin{verbatim}
git commit --amend
git commit --fixup 72856ea      # Marks a commit for later rebase
\end{verbatim}

\subsectiontitle{Interactive rebasing}
\begin{verbatim}
git rebase -i HEAD~5
\end{verbatim}

\end{document}